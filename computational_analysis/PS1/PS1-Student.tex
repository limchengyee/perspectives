\documentclass[letterpaper,12pt]{article}
\usepackage{array}
\usepackage{threeparttable}
\usepackage{geometry}
\geometry{letterpaper,tmargin=1in,bmargin=1in,lmargin=1.25in,rmargin=1.25in}
\usepackage{fancyhdr,lastpage}
\pagestyle{fancy}
\lhead{}
\chead{}
\rhead{}
\lfoot{}
\cfoot{}
\rfoot{\footnotesize\textsl{Page \thepage\ of \pageref{LastPage}}}
\renewcommand\headrulewidth{0pt}
\renewcommand\footrulewidth{0pt}
\usepackage[format=hang,font=normalsize,labelfont=bf]{caption}
\usepackage{listings}
\lstset{frame=single,
  language=Python,
  showstringspaces=false,
  columns=flexible,
  basicstyle={\small\ttfamily},
  numbers=none,
  breaklines=true,
  breakatwhitespace=true
  tabsize=3
}
\usepackage{amsmath}
\usepackage{amssymb}
\usepackage{amsthm}
\usepackage{harvard}
\usepackage{setspace}
\usepackage{float,color}
\usepackage[pdftex]{graphicx}
\usepackage[format=hang,font=normalsize,labelfont=bf]{caption}
\usepackage{hyperref}
\hypersetup{colorlinks,linkcolor=red,urlcolor=blue}
\theoremstyle{definition}
\newtheorem{theorem}{Theorem}
\newtheorem{acknowledgement}[theorem]{Acknowledgement}
\newtheorem{algorithm}[theorem]{Algorithm}
\newtheorem{axiom}[theorem]{Axiom}
\newtheorem{case}[theorem]{Case}
\newtheorem{claim}[theorem]{Claim}
\newtheorem{conclusion}[theorem]{Conclusion}
\newtheorem{condition}[theorem]{Condition}
\newtheorem{conjecture}[theorem]{Conjecture}
\newtheorem{corollary}[theorem]{Corollary}
\newtheorem{criterion}[theorem]{Criterion}
\newtheorem{definition}[theorem]{Definition}
\newtheorem{derivation}{Derivation} % Number derivations on their own
\newtheorem{example}[theorem]{Example}
\newtheorem{exercise}[theorem]{Exercise}
\newtheorem{lemma}[theorem]{Lemma}
\newtheorem{notation}[theorem]{Notation}
\newtheorem{problem}[theorem]{Problem}
\newtheorem{proposition}{Proposition} % Number propositions on their own
\newtheorem{remark}[theorem]{Remark}
\newtheorem{solution}[theorem]{Solution}
\newtheorem{summary}[theorem]{Summary}
%\numberwithin{equation}{section}
\bibliographystyle{aer}
\newcommand\ve{\varepsilon}
\newcommand\boldline{\arrayrulewidth{1pt}\hline}


\begin{document}

\begin{flushleft}
  \textbf{\large{Problem Set \#1}} \\
  MACS 30000, Dr. Evans \\
  Cheng Yee Lim
\end{flushleft}

\vspace{5mm}

\noindent\textbf{Problem 1}

\textbf{Part (a).}\
\begin{figure}[htb]\centering\captionsetup{width=4.0in}
  \caption{\textbf{Screenshot of one simulated lifetime income path}}\label{Fig. 1a}
  \fbox{\resizebox{4.0in}{3.0in}{\includegraphics{images/Fig_1a.png}}}
\end{figure}\

\clearpage
\textbf{Part (b).}\
\begin{figure}[htb]\centering\captionsetup{width=4.0in}
  \caption{\textbf{Screenshot of histogram of 10,000 initial income simulations}}\label{Fig. 1b}
  \fbox{\resizebox{4.0in}{3.0in}{\includegraphics{images/Fig_1b.png}}}
\end{figure}\\\
1.25\% of the class will earn more than \$100,000 in the first year out of the program.\\
9.28\% of the class will earn less than \$70,000 in the first year out of the program.\\
No, the histogram is not normally distributed, it has fatter ends and a higher mean than a normal distribution.\\

\clearpage
\textbf{Part (c).}\

\begin{figure}[htb]\centering\captionsetup{width=4.0in}
  \caption{\textbf{Screenshot of histogram of 10,000 smiulations of years taken to pay off loans}}\label{Fig. 1c}
  \fbox{\resizebox{4.0in}{3.0in}{\includegraphics{images/Fig_1c.png}}}
\end{figure}\

17.95\% of the simulations are able to pay off the loan in 10 years.\\

\clearpage
\textbf{Part (d).}\

\begin{figure}[htb]\centering\captionsetup{width=4.0in}
  \caption{\textbf{Screenshot of histogram of 10,000 smiulations of years taken to pay off loans}}\label{Fig. 1d}
  \fbox{\resizebox{4.0in}{3.0in}{\includegraphics{images/Fig_1d.png}}}
\end{figure}\

74.07\% of the simulations are able to pay off the loan in 10 years, after the Chicago MACSS program becomes very well-known and the skills of the students are in high demand.\\

\end{document}

